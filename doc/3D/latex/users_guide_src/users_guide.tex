\documentclass[11pt]{report}

\oddsidemargin=0.00in
\evensidemargin=0.00in
\topmargin=0.00in
\headheight=0.00in
\headsep=0.00in
\topskip=0.0in
\textwidth=6.50in
\textheight=9.00in
\footskip=0.50in

\title{\Large \bf
       Computational Framework for Fluids and Combustion (CFFC)\\
       \ \\ \Huge
       \protect\underline{CFFC User's Guide}}

\author{\Large \em 
        University of Toronto Institute for Aerospace Studies\\
        \Large \em
        4925 Dufferin Street, Toronto, Ontario, Canada M3H 5T6}

\date{\today}

\begin{document}

\maketitle

\tableofcontents

\chapter{What is {\tt CFFC}?}

{\bf Synopsis}: An overview of the CFFC is given.

\section{The Source Tree}

{\bf Synopsis}: An overview of the source tree

\subsection{CVS}

\subsection{{\tt src} Directory}

\subsection{{\tt bin} Directory}

\subsection{{\tt data} Directory}

\subsection{{\tt doc} Directory}

\section{Makefile}

{\bf Synopsis}: The makefile

\subsection{Support for a Variety of Computer Architectures}

\section{The Development Team}

{\bf Synopsis}: The development team  

%%%%%%%%%%%%%%%%%%%%%%
%% pdes++           %%
%%%%%%%%%%%%%%%%%%%%%%
\chapter{What is {\tt pdes++}?}

{\bf Synopsis}: An overview of the

\section{How to Compile {\tt pdes++}}

{\bf Synopsis}:

\section{How to Run {\tt pdes++}}

{\bf Synopsis}:

\subsection{Command Line Arguments}

\subsection{Input Files}

\section{Examples}

{\bf Synopsis}:

%%%%%%%%%%%%%%%%%%%%%%
%% Solution Classes %%
%%%%%%%%%%%%%%%%%%%%%%
\chapter{{\tt CFFC} PDE Solution Classes}

{\bf Synopsis}: An overview of the

\section{{\tt Euler1D} Class and Supporting Routines}

{\bf Synopsis}: The {\tt Euler1D} class

\section{{\tt Euler2D} Class and Supporting Routines}

{\bf Synopsis}: The {\tt Euler2D} class

\section{{\tt Scalar1D} Class and Supporting Routines}

{\bf Synopsis}: The {\tt EulerD} class

\section{{\tt Heat1D} Class and Supporting Routines}

{\bf Synopsis}: The {\tt Heat1D} class

\section{{\tt Heat2D} Class and Supporting Routines}

{\bf Synopsis}: The {\tt Heat2D} class

\section{{\tt HyperHeat1D} Class and Supporting Routines}

{\bf Synopsis}: The {\tt HyperHeat1D} class

\section{{\tt MHD1D} Class and Supporting Routines}

{\bf Synopsis}: The {\tt MHD1D} class

%%%%%%%%%%%%%%%%%%%%%%
%% Mesh Classes     %%
%%%%%%%%%%%%%%%%%%%%%%
\chapter{{\tt CFFC} Computational Mesh Classes}

{\bf Synopsis}: An overview of the

\section{{\tt Grid2D\_Quad\_Block}: A Multi-Block 
         Two-Dimensional Quadrilateral Mesh Class}

{\bf Synopsis}: The {\tt Grid2D\_Quad\_Block} class

\subsection{Initial Mesh Generation and Mesh Smoothing}

\subsection{Grid Definition Files}

\subsection{Some Default Mesh}

%%%%%%%%%%%%%%%%%%%%%%
%% Block-Based AMR  %%
%%%%%%%%%%%%%%%%%%%%%%
\chapter{Block-Based Adaptive Mesh Refinement}

{\bf Synopsis}: An overview of the 

\section{{\tt AdaptiveBlock2D} Class and Supporting Routines}

{\bf Synopsis}: The {\tt AdaptiveBlock2D} class

\section{{\tt QuadTree} Class and Supporting Routines}

{\bf Synopsis}: The {\tt QuadTree} class

\section{{\tt Octree} Class and Supporting Routines}

{\bf Synopsis}: The {\tt Octree} class

%%%%%%%%%%%%%%%%%%%%%%
%% Parallel & MPI   %%
%%%%%%%%%%%%%%%%%%%%%%
\chapter{Parallel Implementation Via MPI}

{\bf Synopsis}: An overview of the

\section{MPI}

{\bf Synopsis}: The MPI

%%%%%%%%%%%%%%%%%%%%%%
%% NASA Rotors      %%
%%%%%%%%%%%%%%%%%%%%%%
\chapter{The NASA Rotor 37/67 Classes}

{\bf Author}: Tomas Dusatko, dusatko@ecf.utoronto.ca, January 2002\\
\
\noindent
{\bf Synopsis}: The need for detailed, experimental 
flow-field measurements continues to be an important 
issue as computational algorithms evolve towards 
fully three-dimensional viscous solution methods. 
Due to the low number of available experimental data 
sets for turbomachinary flow, NASA undertook the task 
to create several sets of accurate experimental data 
to meet this ongoing demand. Primarily this data can 
be used to validate computational methods for both 2-D 
and 3-D flow scenarios. The purpose of the 
{\tt NASARotor37} and {\tt NASARotor67} classes
are to provide CFD developers with a readily available 
interface to all the geometry and flow data that is 
available for the two NASA axial compressors labelled 
Rotor 37 and Rotor 67. Written in C++, this code can be 
used to process both the experimental and geometric data 
available and generate suitable grids for computation.

This chapter contains a summary of 
the available functions in the {\tt NASARotor37} and 
{\tt NASARotor67} classes and supporting routines and 
their correct usage. The classes include functions that:
(i) return the upstream and downstream boundary conditions 
at any radial location, process experimental flow-field 
data provided by NASA; and (ii) return various parts of the 
rotor geometry for use in other routines, and create single 
and multi-block structured grids for use in CFD computations. 
Also included, is a detailed description of all the 
experimental data that has been made available by the NASA 
Glenn Research Centre.

%%%%%%%%%%%%%%%%%%%%%%%%
%% Supporting Classes %%
%%%%%%%%%%%%%%%%%%%%%%%%
\chapter{Other Supporting Classes and Data Types}

{\bf Synopsis}: An overview of the 

\section{{\tt Vector2D} Class and Supporting Routines}

{\bf Synopsis}: The {\tt Vector2D} class

\section{{\tt Vector3D} Class and Supporting Routines}

{\bf Synopsis}: The {\tt Vector3D} class

\section{{\tt Tensor2D} Class and Supporting Routines}

{\bf Synopsis}: The {\tt Tensor2D} class

\section{{\tt Tensor3D} Class and Supporting Routines}

{\bf Synopsis}: The {\tt Tensor3D} class

\section{{\tt Spline2D} Class and Supporting Routines}

{\bf Synopsis}: The {\tt Spline2D} class

\section{Generic {\tt CFD} Classes and Supporting Routines}

{\bf Synopsis}: Some generic {\tt CFD} class

%%%%%%%%%%%%%%%%%%%%%%%%%
%% Other Documentation %%
%%%%%%%%%%%%%%%%%%%%%%%%%
\chapter{Other Documentation}

{\bf Synopsis}: 

%%%%%%%%%%%%%%%%%%%%%%%%
%% Supporting Classes %%
%%%%%%%%%%%%%%%%%%%%%%%%
\chapter{Questions, Requests, and Feeback}

{\bf Synopsis}: Contact information is given for reaching the
                developers and distributors of the 
                {\tt CFFC} software.

\section{Contact Information}

If you have any questions, request, and/or feedback concerning the 
{\tt CFFC} software or this user's manual,
please contact Prof.\ Clinton Groth at:
%%
\begin{verbatim}
Prof. Clinton P. T. Groth

Phone: (416) 667-7715   Fax: (416) 667-7799
E-mail: groth@utias.utoronto.ca
Home page: http://www.utias.utoronto.ca/~groth

Address: Institute for Aerospace Studies
         University of Toronto
         4925 Dufferin Street
         Toronto, Ontario, Canada M3H 5T6
\end{verbatim}
%%

%%%%%%%%%%%%%%%%%%%%%%%%%
%% Acknowledgments     %%
%%%%%%%%%%%%%%%%%%%%%%%%%
\chapter{Acknowledgments}

The {\tt CFFC} software was developed through 
support received 

\end{document}
